\documentclass{beamer}
\usetheme{Boadilla}

\usepackage{minted}
\usepackage{media9}
\usepackage{hyperref}
\usepackage{listings}


\title{Scoring Neighborhoods on the Earth}
\subtitle{A computational social science project \\based on crowd-sourcing surveys and Elo Rating System.}
\author{Luxin Tian}
\institute{The University of Chicago}
\date{\today}

\begin{document}
\maketitle

\begin{frame}
\frametitle{Overview of the structure}

\begin{columns}
    \column{0.5\textwidth}
        \begin{itemize}
        \item \mintinline{python}{elorating} package
            \begin{itemize}
                \item Implement the Elo Rating algorithm and manage a scoring project. 
            \end{itemize}
        \item \mintinline{python}{pp2_app} module
            \begin{itemize}
                \item Use \mintinline{python}{elorating} package to measure the urban perception of 56 cities around the world. 
            \end{itemize}
        \item \mintinline{python}{baidu_app} module 
            \begin{itemize}
                \item Extend the project to cover mainland China. 
            \end{itemize}
        \end{itemize}

    \column{0.5\textwidth}
        \begin{figure}
            \includegraphics[scale=0.4]{QR.png}
            \centering
        \end{figure}
\end{columns}

\end{frame}

\begin{frame}{Elo Rating System}
\begin{itemize}
    \item An algorithm for calculating the relative skill levels of players in zero-sum games such as chess.
    \item I think it can be used to reveal collective preferences from individual pairwise voting. (See \textit{Social Welfare Functional} in microeconomics)
\end{itemize}

\begin{center}
\begin{tabular}{ c|c|c }
Left & Right & VotingOutcome\\
\hline
 Hyde Park & Kenwood & Left \\ 
 University Park & Pilsen & Right \\
 ... & ... & ...
\end{tabular}
\end{center}

\begin{itemize}
    \item Scoring neighborhoods across countries based on individual pairwise voting  on \textbf{street view images}. 
\end{itemize}
    
\end{frame}

\begin{frame}{\mintinline{python}{elorating} Package for Python}
    \begin{itemize}
        \item Create, add, remove, and query an element. 
        \item Update rating scores based on pairwise competition.
        \item Query the rating score of an element.
        \item Predict winning probability.
        \item Import/export data from/to CSV files. 
        \item Generate descriptive statistics.
        \item Normalize the rating scores to some user-defined scales.
    \end{itemize}

\begin{example}[Install \mintinline{python}{elorating}]
    \mint{shell-session} |>>> pip install /path/to/this/project|
\end{example}


\end{frame}


\begin{frame}{Scoring Neighborhoods in 56 Cities}
    \begin{columns}
    \column{0.5\textwidth}
    \begin{itemize}
        \item Place Pulse 2.0 data 
        \begin{itemize}
            \item A digital survey to humans
            \item Covers 56 cities from 28 countries across 6 continents
        \end{itemize}
        \item Dimensions: 
        \begin{itemize}
            \item Safety
            \item Lively
            \item Wealthy
            \item Beautiful
            \item Depressing
            \item Boring
        \end{itemize}
    \end{itemize}
    
    \column{0.5\textwidth}
    \begin{center}
            \begin{figure}
            \includegraphics[scale=0.4]{QR.png}
            \centering
        \end{figure}
    \end{center}
    \end{columns}
\end{frame}

\begin{frame}{\mintinline{python}{pp2_app}: Calculating Perception Scores}
    \begin{center}
   \includemedia[
  activate=pageopen,
  width=300pt,height=150pt,
]{}{pp2_demo.swf} 
\end{center}
\end{frame}

\begin{frame}{Interactive Maps}
\href{https://luxin-tian.github.io/Scoring-Neighborhoods-on-the-Earth/}{Scoring Neighborhoods on the Earth}
    \begin{center}
       \includemedia[
      activate=pageopen,
      width=300pt,height=150pt,
    ]{}{map_demo.swf} 
    \end{center}
\end{frame}

\begin{frame}{Extend this project to mainland China}

\begin{itemize}
    \item Retrieve street view images within a user-specified geographical area from \href{http://lbsyun.baidu.com/}{Baidu Maps}. 
    \item \mintinline{python}{baidu_app}
    
        \begin{center}
            \begin{figure}
            \includegraphics[scale=0.2]{sample_street_image.jpg}
            \centering
        \end{figure}
    \end{center}
    
    \item In progress... (but all the work in Python has been finished. )
    
\end{itemize}

\end{frame}

\begin{frame}{Reflections}
\begin{itemize}
    \item Challenges
    \begin{itemize}
        \item Structuring the project, organizing multiple modules
        \item Documentation (\mintinline{python}{sphinx})
        \item User Interface
    \end{itemize}
    \item Gains
    \begin{itemize}
        \item Organizing a full Python project
        \item Data processing
        \item Data visualization
        \item Calling APIs
        \item Coding style and documentation (necessary for open-sourcing or collaboration)
    \end{itemize}
    
\end{itemize}
    
\end{frame}

\end{document}